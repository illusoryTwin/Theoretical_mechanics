% \documentclass{article}
% \usepackage{graphicx} % Required for inserting images
% \usepackage{hyperref}
% \usepackage{caption} % For custom captions
% \usepackage{amsmath}
% \usepackage{multicol}

% \title{Theoretical Mechanics: Big Homework 2}
% \author{Ekaterina Mozhegova}
% \date{March 13, 2024}

% \begin{document}

% \maketitle

% \section{Link}
% \href{https://github.com/illusoryTwin/Theoretical_mechanics/tree/master/hw7}{Link to the github repository containing all the materials}


% \section{Task Description}

% \begin{enumerate}
%     \item Obtain the required measurements of the stand (like needed masses, lengths and so on).
%     \item Gather the positions and velocities of the stand. You should run the same experiment 3 times each. 
    
%     Initial conditions:
%     \begin{itemize}
%       \footnotesize
%       \item $x = 0,\ \phi = 15^\circ,\ \dot{x} = 0,\ \dot{\phi} = 0,\ t=0$;
%       \item $x = 0.25\ m,\ \phi = 45^\circ,\ \dot{x} = 0,\ \dot{\phi} = 0,\ t=0$;
%       \item $x = 0.25\ m,\ \phi = -135^\circ,\ \dot{x} = 0,\ \dot{\phi} = 0,\ t=0$;
%     \end{itemize}
%     \item Substitute a real data to your math model from HW 7, 8 (you can choose any method your like) and compare the results (propose and justify the metric).
%     \item Explain, what affects the difference between math model and real data. Is it difference significant? 
%     \item If so, change the model (add new forces, change the object representation), gather new needed data and compare it again.
%     \item Make a conclusion.
% \end{enumerate}

  
% \section{What report should contain}
% \begin{frame}[t]{What report should contain}
% \vspace{-0.4cm}
% \begin{itemize}
% \item The list of used tools and applications (I gathered a trajectory dataset using $x$ tool), etc.
% \item The list of data you gathered from stand and how did you do it.
% \item Show how you conducted experiments. Is there any difference when you did the same experiment? Show it using plots and/or other metrics like $std,\ mse$ and so on.
% \item Show the way how you chose the metrics for trajectory comparison, how you justify the answer.
% \item If the error is too large, explain how you wanted to change your model and why you chose such path.
% \item Summarize your experience.
% \end{itemize}
% \end{frame}
  

% \section{Hints}
% \begin{frame}[t]{Hints}
% \vspace{-0.4cm}
% \begin{itemize}
%     \item To determine inertia, you can create (or find) a CAD model and take data from it, or you can do an actual experiment to determine it.
%     \item When you perform the experiment, you can make a template (for example, a piece of paper with a 10-degree angle).
%     \item For choosing a metric for how to compare trajectories, you can start with the concept of least squares.
%     \item Some of the data for the math model can be taken from the stand itself (e.g., friction).
%     \item Feel free to do a literature review to find good ideas. But you must show exactly what you used.
% \end{itemize}
% \end{frame}

% % \includegraphics*[scale=0.5]{hw7_task1.png}{\centering}
% \subsection{Task Explanation}

% \textbf{Research object:} A system of 2 rods: rod 1 and rod 2\\
% \textbf{Motion:} Rod 1 - rotational and plane motions, rod 2 - rotational and plane motions\\
% \textbf{Force analysis:} $G_1 = m_1 g$, $G_2 = m_2 g$\\
% \textbf{Solution:}

% \[T = A \text{, where } T = \sum T_i, A = \sum A_i\]
% \[T = T_1 + T_2 = 2T_1 \text{(due to the system's symmetry)}\] 
% \[T_1 = \dfrac{m v_d^2}{2} + \dfrac{J \omega^2}{2} = \dfrac{m v_d^2}{2} + \dfrac{m \rho^2 \omega^2}{2}\] (plane motion  and rotational motion)
% \[v_d^2 = \dot y_d^2 + \dot x_d^2\]



% \begin{align*}
%     y_d &= l \sin(\alpha) & \dot{y}_d &= l \dot{\alpha} \cos(\alpha) \\
%     x_d &= l \cos(\alpha) & \dot{x}_d &= -l \dot{\alpha} \sin(\alpha)
% \end{align*}


% \[ \dot \alpha = \omega = \dot \alpha = \dfrac{v_b}{2l \cos \alpha}\]

% \[V_d^2 = (l \cos(\alpha) \dot{\alpha})^2 + (-l \sin(\alpha) \dot{\alpha})^2 = l^2 (\dot{\alpha})^2 = l \omega^2\]

% \[V_d^2 = l^2 \omega^2 = l^2 \cdot \frac{v_b^2}{4l^2 \cos^2(\alpha)}\]

% \[T_\text{rot} = \dfrac{J \omega^2}{2} = \dfrac{m \rho^2 \omega^2}{2} = \dfrac{m \rho^2}{2} \dfrac{V_b^2}{4l^2 \cos(\alpha)^2}\]

% \[T_1 = T + T_{\text{rot}} = \frac{mV_d^2}{2} + \frac{J \omega^2}{2} = \frac{m V_b^2}{2 \cdot 4 \cos^2(\alpha)} + \frac{m \rho^2 V_b^2}{2 \cdot 4l^2 \cos^2(\alpha)}\]

% \[T_\text{tot} = 2T_1 = 2 (\frac{m V_b^2}{2 \cdot 4 \cos^2(\alpha)} + \frac{m \rho^2 V_b^2}{2 \cdot 4l^2 \cos^2(\alpha)}) = \frac{m V_b^2}{4 \cos^2(\alpha)} + \frac{m \rho^2 V_b^2}{4l^2 \cos^2(\alpha)}\]

% \textbf{Work:}

% 2 gravitational forces do job (of a rod 1 and of a rod 2).

% Due to the symmetry:

% \[A_\text{tot} = 2A_G = 2 mg \Delta h = 2mg ( \frac{h}{2} - l \sin(\alpha))\]
 
% \begin{enumerate}
%   \item When B hits the floor, $\alpha = 0$, so $A_\text{tot} = 2 m g \frac{h}{2} = mgh$
%     \begin{center}
%       For $\alpha = 0$ we have: $\cos(\alpha) = 1$, $\sin(\alpha) = 0$.
%     \end{center}
%     \[ \frac{m V_b^2}{4 \cos^2(\alpha)} + \frac{m \rho^2 V_b^2}{4l^2 \cos^2(\alpha)} = mgh\]
%     \[ \frac{m V_b^2}{4 \cdot 1} (1 + \frac{\rho^2}{l^2}) = mgh\]
%     \[ \frac{V_b^2}{4} (\frac{l^2 + \rho^2}{l^2}) = gh\]
%     \[ V_b = \sqrt{\dfrac{4l^2gh}{l^2 + \rho^2}} = 2l \sqrt{\dfrac{gh}{l^2 + \rho^2}}\]
%     \[ \text{So, } V_b = 2l \sqrt{\dfrac{gh}{l^2 + \rho^2}}\]


%   \item When B  is at the distance $ \frac{1}{2} h$ from the floor. 
%     \[\sin(\alpha) = \dfrac{\frac{h}{2}}{2l} = \dfrac{h}{4l}\]
%     \[\text{Thus, } \cos(\alpha)^2 = 1 - \sin(\alpha)^2 = \dfrac{16l^2 - h^2}{16l^2}\]
%     \[ \frac{m V_b^2}{4 \cdot \dfrac{16l^2 - h^2}{16l^2}} (\dfrac{\rho^2 + l^2}{l^2}) = 2mg(\frac{h}{2} - l \sin(\alpha))\]
%     \[ \frac{m V_b^2}{\dfrac{16l^2 - h^2}{4l^2}} (\dfrac{\rho^2 + l^2}{l^2}) = 2mg(\frac{h}{2} - l \frac{h}{4l})\]
%     \[ \frac{4 mV_b^2 (l^2+\rho^2)}{16l^2 - h^2} = \dfrac{mgh}{2}\]
%     \[ \frac{4 V_b^2 (l^2+\rho^2)}{16l^2 - h^2} = \dfrac{gh}{2}\]
%     \[ V_b = \sqrt{\dfrac{gh(16l^2-h^2)}{2 \cdot 4(l^2 + \rho^2)}} = \dfrac{1}{2} \sqrt{\dfrac{gh(16l^2 - h^2)}{2(l^2 + \rho^2)}}\]
%     \[ \text{So, } V_b = \dfrac{1}{2} \sqrt{\dfrac{gh(16l^2 - h^2)}{2(l^2 + \rho^2)}}\]
% \end{enumerate}

% \textbf{Answer:}
% \begin{enumerate}
%   \item \[V_b = 2l \sqrt{\dfrac{gh}{l^2 + \rho^2}}\]
%   \item \[V_b = \dfrac{1}{2} \sqrt{\dfrac{gh(16l^2 - h^2)}{2(l^2 + \rho^2)}}\]
% \end{enumerate}

% \section{Task 2}

% \section{Task Description}

\documentclass{article}
\usepackage{graphicx} % Required for inserting images
\usepackage{hyperref}
\usepackage{caption} % For custom captions
\usepackage{amsmath}
\usepackage{multicol}

\title{Theoretical Mechanics: Big Homework 2}
\author{Ekaterina Mozhegova}
\date{March 13, 2024}

\begin{document}

\maketitle

\section{Link}
\href{https://github.com/illusoryTwin/Theoretical_mechanics/tree/master/hw7}{Link to the GitHub repository containing all the materials}

\section{Task Description}
\begin{enumerate}
    \item Obtain the required measurements of the stand (like needed masses, lengths and so on).
    \item Gather the positions and velocities of the stand. You should run the same experiment 3 times each. 
    
    Initial conditions:
    \begin{itemize}
      \footnotesize
      \item $x = 0$, $\phi = 15^\circ$, $\dot{x} = 0$, $\dot{\phi} = 0$, $t=0$;
      \item $x = 0.25\ \text{m}$, $\phi = 45^\circ$, $\dot{x} = 0$, $\dot{\phi} = 0$, $t=0$;
      \item $x = 0.25\ \text{m}$, $\phi = -135^\circ$, $\dot{x} = 0$, $\dot{\phi} = 0$, $t=0$;
    \end{itemize}
    \item Substitute real data to your math model from HW 7, 8 (you can choose any method you like) and compare the results (propose and justify the metric).
    \item Explain what affects the difference between the math model and real data. Is the difference significant? 
    \item If so, change the model (add new forces, change the object representation), gather new needed data and compare it again.
    \item Make a conclusion.
\end{enumerate}

\subsection{What report should contain}
\begin{itemize}
\item The list of used tools and applications (I gathered a trajectory dataset using $x$ tool), etc.
\item The list of data you gathered from the stand and how did you do it.
\item Show how you conducted experiments. Is there any difference when you did the same experiment? Show it using plots and/or other metrics like $std,\ mse$ and so on.
\item Show the way how you chose the metrics for trajectory comparison, how you justify the answer.
\item If the error is too large, explain how you wanted to change your model and why you chose such a path.
\item Summarize your experience.
\end{itemize}

\subsection{Hints}
\begin{itemize}
    \item To determine inertia, you can create (or find) a CAD model and take data from it, or you can do an actual experiment to determine it.
    \item When you perform the experiment, you can make a template (for example, a piece of paper with a 10-degree angle).
    \item For choosing a metric for how to compare trajectories, you can start with the concept of least squares.
    \item Some of the data for the math model can be taken from the stand itself (e.g., friction).
    \item Feel free to do a literature review to find good ideas. But you must show exactly what you used.
\end{itemize}

\section{Task Explanation}

\textbf{Research object:} A system of 2 rods: rod 1 and rod 2\\
\textbf{Motion:} Rod 1 - rotational and plane motions, rod 2 - rotational and plane motions\\
\textbf{Force analysis:} $G_1 = m_1 g$, $G_2 = m_2 g$\\
\textbf{Solution:}

\section{Plots}
\section{Simulation}

\end{document}